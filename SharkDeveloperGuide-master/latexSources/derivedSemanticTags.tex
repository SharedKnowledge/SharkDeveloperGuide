\chapter{Derived Semantic Tags}
Semantic Tags describe things in the real world. In academic speech: They are a {\it representation} of a real thing in a computer system.

Anything can be represented with Semantic Tags due to its general concept. Three special types of {\it things} are of higher importance in Shark which are: Peers (persons in most cases), locations and time frames.

A special Semantic Tag type exists for each of them.

\section{Peer Semantic Tags}
A {\it Peer Semantic Tag (PST)} represents a peer. We have already learned that a peer can represent a person, a group of persons or a mixture of both.

Shark only assumes that a peer can communicate with other peers. Thus, a peer can have addresses which can be e-mail addresses, TCP addresses etc. Shark supports e-mail and TCP in version 2.0. Each address can be represented with a string. Each address type starts with a prefix:

\begin{description}
    \item[mail://] indicates that a e-mail address follows. mail://alice@wonderland.net is a valid address in Shark. Note: The prefix {\bf is not} {\tt mailto} which is used in HTML. 

A mail address can have an additional parameter: {\it maxLength}. It describes the maximal message length in kilo bytes that fits into the mailbox. 

mail://alice@wonderland.net?maxlLength=2048 is a valid address. It describes the fact that must not exceed the length of 2 MByte\footnote{It can also describe that users don't want to receive message larger than the defined length. The given parameter won't be compared to the information given by the mail provider.}.

If no {\it maxLength} is defined, the default value is used, which is defined as one MByte.

    \item[tcp://] indicates that a TCP address follows.
tcp://shark.wonderland.net:7070 is a valid address. A peer can start a TCP server and communicate its address to other peers\footnote{It is highly recommended for any developer to make detailed security considerations before starting a TCP server on their machines. It might not be the best idea to write applications that present an open TCP port to the Internet without any kind of intrusion detection or application firewall.}.

\end{description}

A Peer Semantic Tag simply adds an arbitrary number of (permanent or temporary) addresses to a Semantic Tag. For example a peer named Alice could be described with the following PST: 

\begin{description}
    \item[Name] Alice.
    \item[Subject Identifier(s)] http://www.sharksystem.net/alice.html
    \item[Address(es)] mail://alice@wonderland.net
\end{description}

It is straightforward and so is the code:

\begin{verbatim}
String aliceSI = "http://www.sharksystem.net/alice.html";

String aliceMail = "mail://alice@wonderland.net";
String aliceTCP = "tcp://shark.wonderland.net:7070";

String[] aliceSIs = new String[] {aliceSI};
String[] aliceAddr = new String[] {aliceMail, aliceTCP};

PeerSemanticTag alice = 
  InMemoSharkKB.createInMemoPeerSemanticTag("Alice", 
    aliceSIs, aliceAddr);

System.out.println(L.semanticTag2String(alice));
\end{verbatim}

\section{Time Semantic Tags}
Time semantic tags describe a period of time. They are mainly used to describe time frames in which interests or information offerings are valid. Interests and information are discussed later in this book.

Periods can be described by a start and duration. Both values have to be defined with milliseconds.

The following code creates a TST that covers a whole day beginning {\it now}.

\begin{verbatim}
long now = System.currentTimeMillis();

long dayLength = 24 * 60 * 60 * 1000; 

TimeSemanticTag time = 
    InMemoSharkKB.createInMemoTimeSemanticTag(now, dayLength);

System.out.println(L.semanticTag2String(time));
\end{verbatim}

The length of a day is calculated. Each day has 24 hours with 60 minutes. Each minute has 60 seconds which has 1000 milliseconds.

\section{Spatial Semantic Tags}
Information or information request can be location based. We need means to
describe locations. There is a quite simple data model called {\it simple feature model} which is supported by nearly any spatial database. That
model is based on simple geometric structure, namely points, lines and polygones even with wholes. Arbitrary geometries can be combined with a collection which is also a geometry.

The {\it well known text (WKT)} defines a string format to define those geometries.

Shark supports WKT. Of course, Shark has to integrate spatial information into it general model. Spatial Semantic Tags (SST) are created for that reason. 
SST add a geometry to a semantic tag.

Thus, a SST can be created with name, subject identifier and geometry.
A geometry can be defined with a valid WKT string.

Moreover, geometries can also be defined with absolute values.
The following code creates a point with GPS coordinates of a point in Berlin.
Afterwards, a SST is created which has no semantics but a geometry.

\begin{verbatim}
Geometry geom = 
  InMemoGeometry.createPoint(52.45747, 13.52669);

SpatialSemanticTag berlinHTW = 
  InMemoSharkKB.createInMemoSpatialSemanticTag(geom);

System.out.println(L.semanticTag2String(berlinHTW));
\end{verbatim}

Note, Shark only supports GPS coordinates! More specific, Shark only supports geometries with EPSG code 4326. Shark is - not yet - a spatial database. But it links spatial data with semantics.

\subsection{Summary}
In general, we strongly discourage usage of {\it any} tags. We can weaken that rule for time and spatial tags. Time semantic tags describe a period and 
take their identity, their meaning exactly from that definition. 
Spatial geometries have already their own meaning. They define an area on earth. That can be sufficient sometimes. Sometimes it might help to add more semantics. It depends on the applications. We come back to that point when we discuss interests.

\subsection{Exercises}
\begin{enumerate}
\item 
Describe yourself as peer semantic tag.
\item 
Define your lifespan as time semantic tag.
\item 
Define your place of birth as spatial semantic tag.
\end{enumerate}
