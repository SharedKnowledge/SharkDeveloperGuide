\chapter{SharkKB implementations}
\label{sec:sharkkbimplementations}
This will be a very brief chapter. Implementations of {\tt SharkKB} are to be discussed. There are two implementations in version 2.0. The {\tt InMemoSharkKB} keeps all data in memory and offers no persistence. Knowledge base methods are already discussed. There are no special in memory methods.

{\tt FSSharkKB} derives from its in-memory-parent and keeps all the data in the file system. In most cases, data is taken from files and kept in memory. Yet, there are no more or less sophisticated swapping techniques implemented. That's up to future extensions.

Creating a file system based knowledge base is simple. Just name a folder and create it:

\begin{verbatim}
String folderName = "mySharkKB";
SharkKB fsKB = new FSSharkKB(folderName);
\end{verbatim}

In this case, a folder named {\it mySharkKB} is created in the working directory if it does not exist yet. All data is stored in this folder and its sub-folders. 

We are not going to explain the folder and file structure for two reasons. Some - but not you, dear reader - would come up with the insane idea of using these files directly to do some things {\it faster} or {\it better}. Please, never ever try to even think of things like that. Keep that folder and its content as terra incognita. Yes, the whole folder can be moved. It can even be zipped and sent e.g. via e-mail to others. But never ever edit it without the framework.

The second reason is simple: Maybe, we decide to change the folder structure. We'd have to rewrite a whole chapter. Who really likes these jobs? I don't.

There is one special method in this class that should be explained:

\begin{verbatim}
String freshFileName = FSSharkKB.emptyFilename("proposedName");
\end{verbatim}

This methods checks if a folder with the {\it proposedName} already exists. {\it ProposedName} itself is returned if no such folder exists. If a folder exists, a number is added and the algorithm is performed again and again until a not-existing folder name was found. This method can be used to ensure that no other folder is accidentally overwritten by our new knowledge base.
