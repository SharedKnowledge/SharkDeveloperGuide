\chapter{Security}
\label{sec:security}
Shark allows encrypting and signing of KEP messages. We explain the responsible components in this chapter but don't explain in detail underlying mechanisms.

\section{Signing}
Signing a message requires a private key. A hash code of a message is created and encrypted with a private key. That encrypted hash code (which is called signature) is added to the message.

A recipient can read the message and calculate the same hash code. Recipient can decrypt the transmitted signature with senders public key. The signature is meant to be {\it verified} if both hash codes are equal.

Verifying requires a safe and secure way for public key dissemination. Such an infrastructure is called Public Key Infrastructure (PKI). Shark can be used with any PKI\footnote{There is already a Shark PKI that needs more testing and becomes part of version 3.0 of Shark.}. Shark uses MD5 for hash code calculation and RSA for signature creation and verifying.

\section{Encrypting}
Messages can be encrypted with RSA in Shark. The recipients public key is used to encrypt the message. The private key is used to decrypt the message. Assumed, the private key is not compromised, the message can only be read by the recipient.

Apparently, message encryption requires public keys as well. Thus, a PKI is needed, see above.

Signing and encrypting can be switched on and off in Shark. We strongly suggest using both if a PKI is available. In Shark, we have chosen to encrypt messages before signing it. Thus, a message signature is verified before the 
actual message is decrypted.

\section{Shark Key Store}
Signing and encryption requires a key pair. Each peer has its own private key and needs a way to get a public key from another peer. Keys must be stored. Such stores are usually and less surprisingly called {\it key stores}.

Shark defines feature of a key store in the {\tt SharkPublicKeyStorage} interface since Shark 2. There is a single implementation based on files system ({\tt FSSharkKBPublicKeyStorage}). Developers can decide to use another key store. A wrapper\footnote{A class that implements {\tt SharkPublicKeyStorage} and delegates calls to the real underlying key store.} is required to attach the key store to Shark.

The following code illustrates the core feature of key store.

\begin{verbatim}
// kb to store keys
FSSharkKB baseKB = new FSSharkKB("testKB");

//each store needs an engine - WHY IS THAT?
J2SEAndroidSharkEngine se = new J2SEAndroidSharkEngine();

// create storage
SharkPublicKeyStorage ks = new FSSharkKBPublicKeyStorage(baseKB, se);

// create or get private key;
PrivateKey privateKey = ...;

// save with key store
ks.setPrivateKey(privateKey);

// define a peer
PeerSemanticTag peer = ...;

// assume that's peers public key
PublicKey publicKey = ...;

// assume we have signed that public key
byte[] signature = null;

// create a structure
SigningPeer signingPeer = new SigningPeer(peer, signature);

long valid = 1000 * 60 * 24 * 12 * 365; // 365 Tage
ks.addPublicKey(publicKey, peer, signingPeer, valid);

// retake that structure from key store
SharkCertificate certificate = ks.getCertificate(peer);

// get private key from key store (there is only one)
ks.getPrivateKey();
\end{verbatim}

The peer is identified by one or more subject identifiers - as usual in Shark.
It is up to application developers to implement that interface and attach a PKI.

We hope the code above is self-explaining. A key store is created with a knowledge base and an engine. Keys can be stored and retrieved. Public keys can also be signed. The actual signing algorithm is up to the developers. Java offers a number of methods to sign public keys. A signature combined with its key makes up a certificate. Thus, public key can be retrieved in its certificate incarnation from the key store.

\section{Setting up security policy}
Shark can sign and encrypt but it cannot in any case: Signing requires a private key. Encrypting requires recipients public key. Signature verifying requires senders public key. Moreover, how should a peer reply to a received message. Shall it sign and encrypt in any case? Shall is use the same method as its communication partner? 

Shark cannot make that decision. It is up to the application. Shark provides security policies which can be set. That's just a single line of code.

Developers can define if encryption and/or signing is to be used. They can chose between three options.
\begin{description}
    \item[MUST] defines that the usage of encryption or signing is not optional. This setting ensures that no issued message is unencrypted and unsigned. 

Sending a message can fail if no private key was set. In this case, signing is not possible and the message won't be sent. Sending a message can fail if the recipients public key cannot be found. The encryption cannot be fulfilled in this case and no message is sent.

    \item[IF\_POSSIBLE] is similar to must: The message is signed if a private key is available. The encryption is done if a public key can be found. The messages are sent unsigned or unencrypted if a key is missing.

    \item[NO] defines that a message is not encrypted and/or signed.
\end{description}

\subsection{Reply policies}
The previous setting described how initial messages shall be sent. Peers can receive messages. There are more settings which describe how to react to a message with some security settings.

There are three options as well.

\begin{description}
    \item[TRY\_SAME] A message that reaches a peer can be signed or not and encrypted or not. A peer will try to use the same security means. It will e.g. try to send a signed message if it received a signed message. It will try to send an unsigned message if the received message was unsigned, though.

The attempt can fail. Signing requires a private key and encryption a public key. Thus, signing and/or encryption can fail if the peer lacks of either key. Such failures are ignored and the message is sent unsigned and not encrypted.

    \item[SAME] This message is a variant of {\tt TRY\_SAME} with a more harsh reaction in case of a failure. A reply won't be sent if it cannot use the same security policy as the received message. The sending of the message will fail in that case which means that no message is sent.

Apparently, sending a reply to an unsigned and unencrypted message cannot fail due to security reasons.

    \item[AS\_DEFINED] The third option defines that the peer ignores any security setting of an incoming message and uses the security level as it was defined for messages originated by the peer itself.

\end{description}

Security settings are defined with the {\tt initSecurity} method on a shark engine.

\begin{verbatim}
// an engine was created sometimes..
SharkEngine aliceSE = ...;

//security levels are set
aliceSE.initSecurity(
   this.meAlice, // define owner of key store and private key
   aliceKeyStore, // the actual keystore
   SharkEngine.SecurityLevel.NO, // encryption (NO in this case)
   SharkEngine.SecurityLevel.NO, // signing level (NO in this case) 
   SharkEngine.SecurityReplyPolicy.AS_DEFINED, // reply policy
   false // don't refuse signed message that cannot be verified
);
\end{verbatim}

The previous settings are also the default settings. Engine with no
explicit settings won't sign and encrypt messages. The [receiver?] can't verify signatures either due to the lack of a key store that provides public keys.

Developers don't have to deal directly with a key store if a public key infrastructure is in place.

\section{Shark Public Key Infrastructure}
Shark 3.0 will provide its own PKI. Stay tuned.


