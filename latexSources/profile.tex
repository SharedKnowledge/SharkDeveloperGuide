\section{Profile}
\label{sec:knowledgePorts:Profile}
Peers like Alice and Bob sometimes have more relevant information about themselves. If there is more than temporal interests an instance to store and collect theses relevant facts about peers is needed. Therefore we can use the profile. A profile can change its shape so it fits the needs. E.g. its possible to make an entry "ProfileName" and these entry can store a first name, a last name and a nick name but the entry can easily be changed so if there is no need for a nick name it can be changed into alternate spellings for the first name or add several new components to the "ProfileName" like academical titles and so on. 
An entry is a generic component of a profile. A profile can handle/manage several different entries. Content stored in an profile entry is persistent. 

\subsection{ProfileFactory}
For creating/rebuilding a profile you always use the profile factory. These factory is like the manager of all profiles in the SharkKB. It provides functions to rebuild profiles out of context points and a function to create an empty new profile object {\tt createProfile(PeerSemanticTag, PeerSemanticTag)}. The first PeerSemanticTag parameter is called creator and the second PeerSementicTag parameter is called target. Creator is a peer that makes a new profile and maintains it and a target is a peer all information of the profile refers to. E.g. a profile creator writes a profile over a target. A profile factory needs to be initialized with a SharkKB. All profiles that are created with this factory are now stored in the KB. But the factory can only rebuild profiles from this specific SharkKB so if there is no such profile in the KB an SharkKBException is thrown. To rebuild a profile from an other KB create a new profile factory and initialize it with the KB.

\subsubsection{Creating a profile}
\begin{verbatim}
public class CreateProfile {
	//Create an empty SharkKB
    private SharkKB kb = new InMemoSharkKB();
    
    //Create a profile factory
    private ProfileFactory profileFactory;
    
    //Create Alice addresses
    private String[] aliceAddresses = {"mail://alice@wonderland.net", "mail://alice@wizards.net", "http://www.sharksystem.net/alice.html"};
    
    //Create Alice semantic identifiers 
    private String[] aliceSis = {"http://www.sharksystem.net/alice.html"};
    
    //Create Alice as peer
    private PeerSemanticTag alice = InMemoSharkKB.createInMemoPeerSemanticTag("Alice",
            aliceSis,
            aliceAddresses);
            
    //Initialize profileFactory with SharkKB
    private CreateProfile() throws SharkKBException {
        profileFactory = new ProfileFactoryImpl(kb);
    }

    public Profile createProfileAlice() throws SharkKBException {
    	//Create profile for Alice with the profileFactory
        Profile aliceProfile = profileFactory.createProfile(alice, alice);
        //Create ProfileName object
        ProfileName profileName = new ProfileNameImpl("Alice");
		//Fill ProfileName with data
        profileName.setLastName("Alpha");
        profileName.setTitle("Prof.");
		//Set the ProfileName in Alice profile
        aliceProfile.setName(profileName);
        return aliceProfile;
    }
\end{verbatim}

In these class a profile factory is used to create an profile for Alice. At first there is an empty SharkKB created which initialized the profileFactory, than a peer for Alice is created.
This peer is needed because the createProfile function of the profileFactory needs peers as parameters. "aliceProfile" and a "profileName" object are created. The "profileName" object was filled with information about Alice first name, last name and titles of Alice. "setName" is called on "aliceProfile" and sets the "profileName" object persistently for Alice profile.
A profile consists of all the entry functionality and some basic functions. These basic functions store data which often occurs in a normal profile like:
\begin{verbatim}
	setName(ProfileName profileName) 
	setPicture(byte[] content, String contentType, String identifier)
	setTelephoneNumber(String number, String identifier)
\end{verbatim}
Just 3 basic functions to store often occurring data. It is possible to create exactly these 3 basic functions of the profile with the entry functionality but in terms of simplification they are given by the profile.

\subsubsection{Getting profiles}
\begin{verbatim}
public class GettingProfiles {
    public static void main(String[] args) throws SharkKBException {
    	//Create empty SharkKB
    	SharkKB kb = new InMemoSharkKB();
    	ProfileFactory profileFactory = new ProfileFactoryImpl(kb);
    	
    	//Create a peer semantic tag describing Alice herself
    	String[] aliceAddresses = {"mail://alice@wonderland.net", "mail://alice@wizards.net", "http://www.sharksystem.net/alice.html"};
    	String[] aliceSis = {"http://www.sharksystem.net/alice.html"};
    	PeerSemanticTag alice = InMemoSharkKB.createInMemoPeerSemanticTag("Alice", aliceSis, aliceAddresses);
    	
    	//Create a peer semantic tag describing Bob himself
    	String[] bobAddresses = {"mail://bob@sharknet.net", "mail://bob@wizards.net", "http://www.sharksystem.net/bob.html"};
    	String[] bobSis = {"http://www.sharksystem.net/bob.html"};
    	PeerSemanticTag bob = InMemoSharkKB.createInMemoPeerSemanticTag("Bob", bobSis, bobAddresses);
    	
    	//Profile is created where Alice is the creator and the target
    	profileFactory.createProfile(alice, alice);
    	
    	//Profile is created where Bob is the creator and the target
    	profileFactory.createProfile(bob, bob);
    	
    	//Create a list with Alice and Bobs profile in it
		List<profile> aliceAndBobProfiles = profileFactory.getAllProfiles();  	
		
		//Getting Alice profile out of the profile factory respectively out of the KnowledgeBase
		Profile aliceNew1 = profileFactory.getProfile(alice);
		Profile aliceNew2 = profileFactory.getProfile(alice, alice);
		
		//aliceNew1 and aliceNew2 are equal profiles of Alice
    }
}
\end{verbatim}

The first part is very similar to the example CreateProfile, because before extracting or getting profiles from the factory they must be created and in this example an empty SharkKB is used so it needs to be filled with profiles before. After the create section the SharkKB in the profileFactory now holds a profile for Bob and a profile for Alice. These profiles can be extracted by using these functions: {\tt List<Profile> getAllProfiles()}, {\tt Profile getProfile(PeerSemanticTag creatorAndTarget)} or {\tt Profile getProfile(PeerSemanticTag creator, PeerSemanticTag target)}. The first function returns a list with all profiles in the given profile factory, the second function needs a PeerSementicTag as parameter and will return a profile where the given peer is creator and target and the last function needs two PeerSemanticTags as parameters and will return a specific profile where the first peer of the profile is creator and the second peer of the profile is target.

\subsection{Entry}
An entry is an approach to be a generic container for information. Different kinds of information should be stored in entries. So the structure of entries is also alterable. 
Entry Example: Imagine an entry like a job employment. A job employment can contain things like:
 -employer name(String)
 -job title(String)
 -start and end of the job(Date or Int)
 -information if the job is the current job(Boolean)
 -job description(String)
So now you can create an entry named employment and add the different paths with the information as content.

\subsubsection{Create employment entry}
\begin{verbatim}

\end{verbatim}

















































